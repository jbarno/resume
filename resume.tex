\documentclass[margin,line]{res}

\usepackage{hyperref}
\hypersetup{colorlinks=True, urlcolor=cyan}

\usepackage{multicol}
\oddsidemargin -.5in
\evensidemargin -.5in
\textwidth=6.0in
\itemsep=0in
\parsep=0in

\newenvironment{list1}{
  \begin{list}{$\enumerate$}{
      \setlength{\itemsep}{0in}
      \setlength{\parsep}{0in} \setlength{\parskip}{0in}
      \setlength{\topsep}{0in} \setlength{\partopsep}{0in} 
      \setlength{\leftmargin}{-0.3in}}}{\end{list}}
\newenvironment{list2}{
  \begin{list}{$\bullet$}{
      \setlength{\itemsep}{0in}
      \setlength{\parsep}{0in} \setlength{\parskip}{0in}
      \setlength{\topsep}{0in} \setlength{\partopsep}{0in} 
      \setlength{\leftmargin}{0.2in}}}{\end{list}}
\newenvironment{list3}{
  \begin{list}{$\description$}{
      \setlength{\itemsep}{0in}
      \setlength{\parsep}{0in} \setlength{\parskip}{0in}
      \setlength{\topsep}{0in} \setlength{\partopsep}{0in} 
      \setlength{\leftmargin}{0in}}}{\end{list}}

\begin{document}

\name{Justin Howard Bolden Barnowski \vspace*{1.35cm}}
% HACK: res.cls has listing for addresses - use instead to show current
% position
\address{ 5467 S Ingleside Ave}
\address{ APT 3W, Chicago IL} 

\begin{resume}

\section{\sc Contact Information}
\begin{tabular}{@{}p{3in}p{4in}}
{\bf Phone:  } (847) 946 - 4425\\
{\bf E-mail: } {\tt jbarno54@gmail.com} \\
{\bf Website:} \url{https://github.com/jbarno} \\
\end{tabular}


\section{\sc Education}
  {\bf University of Michigan}, Ann Arbor, MI \hfill {\bf Aug 10 - Dec 13}\\
  {\em B.S.E. Computer Science}


\section{\sc Employment}

\begin{list1}

\item [] {\bf University of Chicago, Center for Data Intensive Science}, Chicago, IL\\
\item [] {\em Software Engineer in Test Genomic Data Commons} \hfill {\bf June 17 - Present}

\begin{enumerate}
\item Making the auto\_qa repository into a functional service. We can now deploy auto\_qa as a service within
a cluster to validate that particular cluster. We are able to now run chrome headlessly both locally and on
an auto\_qa service. Additionally, these tests are able to be scheduled at regular intervals. Before this was
a manual process to run auto\_qa locally while connected to a VPN to validate that given environment. We
now have the ability to conditionally execute tests which require authorization tokens. Finally, auto\_qa had
26 runnable tests when I started there are now {\begin{verbatim}245\end{verbatim}} tests. At least 150 of those were written by me.
\item Working within the gdcapi for both “unit” and end to end tests. Refactored the gdcapi unit tests to to
maintain consistent code coverage with reduced execution time (on TravisCI from ~30 min to ~8 min).
This allows developers to move faster with the confidence that their changes did not break anything.
Automated (12/20) of the API regression tests by translating curl calls to simple
python tests utilizing the requests library. These tests run daily, and will free up one manual tester for
about half a day per release cycle. Identified pain points within the gdcapi around BAM slicing requests
producing unnecessary crosstalk through consul to other gdcapi machines.\\
\end{enumerate}

\item [] {\bf Tophatter Inc}, Palo Alto, CA\\
\item [] {\em Developer} \hfill {\bf June 16 - Feb 18}

\begin{list2}
\item  Upgraded Android application’s user interface to Google’s Material Design specifications.
\item  Orchestrated A/B tests to determine optimal user experience and fulfillment of Tophatter’s objectives.
\item  Learned Ruby to fix server-side bugs while also developing new features on the Android client.
\item  Extended the Android application internationally, specifically enabling multi-currency bidding.
\item  Libraries/Tools used while at Tophatter: databinding, crashlytics, eventBus, picasso, PubNub, FB and Google signup, and braintree payments.\\
\end{list2}

\item [] {\bf Ericsson Inc}, Santa Clara, CA\\
\item [] {\em Developer} \hfill {\bf June 15 - June 16}

\begin{list2}
\item Maintained and enhanced the web app half of a multi-platform video streaming application (Similar to AT\&T U-Verse or Comcast Xfinity).
\item Augmented landing service, and multi-operator customization framework. Including enabling configurable Content-Security-Policy headers to better secure end users’ browser sessions.
\end{list2}

\item [] {\em Tester} \hfill {\bf March 14 - June 15}

\begin{list2}
\item Implemented automated tests using a selenium web driver in both C\# and javascript.
\item Scrum Master for multiple teams.
\end{list2}

\item [] {\bf Microsoft Inc}, Mountain View, CA\\
\item [] {\em SDET Intern} \hfill {\bf May 13 - July 13}

\begin{list2}
\item During a twelve week internship collaborated with the Test Tools and Infrastructure team to revamp failing test case triage process. Worked extensively with Team Foundation Server APIs.\\
\end{list2}

\item [] {\bf Garmin Inc}, Novi, MI\\
\item [] {\em SDET Intern} \hfill {\bf April 12 - Aug 12}

\begin{list2}
\item Modernized the semi-truck instrument cluster, radio, and navigation tools. Worked extensively with the J1939 CAN networks of the vehicle. Including mimicking a CAN network using EEPod’s MyCanic.
\end{list2}

\end{list1}

% \section{\sc Honors and Awards}
% \begin{tabular}{@{}p{3in}p{4in}}
% Branstrom Award Top 5 GPA of Freshmen              & {\bf March 11}\\
% Graduated Cum Laude GPA   3.329                    & {\bf Dec 13}\\
% HKN (Eta Kappa Nu)                                 & {\bf Fall 11}\\
% \end{tabular}

\section{\sc Professional Experience}

\begin{list1}
\item [] {\bf Black Hat}, Las Vegas, NV\\
\item [] {\em Student/Participant } \hfill {\bf Aug 14-15}\\
\begin{list2}
\item Android Application Hacking
\item Uses and Misuses of Cryptography
\item Advanced Wi-Fi Penetration Testing
\item 0x7DF Mobile Application Bootcamp

\end{list2}
\end{list1}


\section{\sc Virtual}
\begin{list1}
\item [] {\bf Android Developement}
\begin{description}
\item Tophatter \hfill {\bf Android Java}
\item Ericcson \hfill {\bf Multi-Client Shell Interface}
\item Trend Phrase \hfill {\bf  UI Class}
\item Stock Stalkers \hfill {\bf Good idea bad Parse}
\item Friend Finder Yahoo’s Hack-U competition \hfill {\bf Fall 10}
\end{description}
\end{list1}
%Currently developing a text based Android game.


\section{\sc Real}

\begin{list1}
\item [] Running
\begin{description}
  \item Chicago Half Marathon\hfill {\bf  Sep 24th, `17 in 2:29:17 }
  \item Chicago Marathon,{\em  The Team to End AIDS} \hfill {\bf  Oct 9th, `11 in 5:35:42}
% Raised ~\$1,700 for Chicago Aids Foundation. Completed my first
% marathon after eight months of self-directed training.
\end{description}
\end{list1}

\section{\sc Buzz Words}
\begin{multicols}{3}
Languages
\begin{list2}
\item C++
\item Python
\item Java
\item Go
\item Typescript
\item Web Developement
\item \LaTeX
\item R
\item Erlang
\item Kotlin
\item C\#
\item Ruby
\end{list2}
\columnbreak
Envs
\begin{list2}
\item Consul
\item linux
\item Mac
\item OpenStack
\item Azure
\end{list2}
\columnbreak
People Skills
\begin{list2}
\item Agile Development
\item Scrum Master
\end{list2}
\end{multicols}

\end{resume}
\end{document}